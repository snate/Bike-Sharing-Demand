\section{Introduzione}

\subsection{Abstract}
Il progetto consiste nell'analisi di dati riguardanti il \emph{Bike sharing}.
Il \emph{Bike sharing} è un servizio che viene fornito nelle principali città,
tramite il quale è possibile utilizzare una bici prelevandola da un chiosco
multimediale e riportandola ad un altro chiosco una volta terminato l'utilizzo.

In questo modo le persone possono spostarsi per la città in base ad esigenze
impreviste, senza aver doversi muovere con la propria bici, cosa
particolarmente scomoda se per raggiungere il posto in cui serve
effettivamente utilizzare la bici bisogna utilizzare un servizio pubblico come
autobus o treno.

Naturalmente, questi spostamenti possono risultare molto interessanti per i
\emph{data scientist} poichè (ad esempio) forniscono dati su:

\begin{itemize}
\item durata del viaggio;
\item luoghi di partenza più ricorrenti;
\item luoghi di arrivo più ricorrenti.
\end{itemize}

Questa possibilità di ottenere dati in un modo molto semplice ha spinto gli
stessi ricercatori ad analizzare la richiesta del servizio di \emph{Bike
sharing}, così da poter prevedere le possibili quantità di dati disponibili.

\subsection{Dati utilizzati}\label{sec:intro-dati}

I dati utilizzati provengono da un contest su
Kaggle\footnote{\texttt{https://www.kaggle.com/c/bike-sharing-demand}},
che a sua volta li ha recuperati dal repository del corso di
\emph{Machine Learning} dell'\textbf{University of California, Irvine}.

Sono disponibili due set, uno per l'addestramento (che chiameremo anche
\emph{training set}) e uno (il \emph{test set}) su cui fare previsioni
utilizzando il modello ``migliore'' ottenuto grazie all'analisi del training
set.

Tuttavia, il test set non verrà utilizzato poichè esso non contiene i valori
``veri'' delle variabili risposta e da qui in poi l'unico dataset a cui ci
riferiremo sarà il \textbf{training set}.

Il dataset fornito presenta le seguenti variabili:

\begin{itemize}
\item \textbf{Variabili esplicative}:
  \begin{itemize}
  \item \texttt{datetime}: variabile che presenta valori del tipo ``YYYY-MM-AA
    HH:MM:SS'', ovvero la data in formato americano seguita dall'orario
    giornaliero

  \item \texttt{season}: variabile qualitativa che assume i seguenti valori:
    \begin{itemize}
    \item \textbf{1}: primavera
    \item \textbf{2}: estate
    \item \textbf{3}: autunno
    \item \textbf{4}: inverno
    \end{itemize}

  \item \texttt{holiday}: variabile qualitativa che può valere \textbf{1}
  (giorno festivo) o \textbf{0} (giorno non festivo):

  \item \texttt{working day}: variabile qualitativa che può valere \textbf{1}
  (giorno feriale) o \textbf{0} (giorno non feriale):

  \item \texttt{weather}: variabile qualitativa che assume i seguenti
    valori:
    \begin{itemize}
    \item \textbf{1}: terso, con poche nuvole o parzialmente nuvoloso
    \item \textbf{2}: nebbia con possibili nuvole
    \item \textbf{3}: pioggia
    \item \textbf{4}: temporale, neve o grandine
    \end{itemize}

  \item \texttt{temp}: temperatura reale in gradi Celsius

  \item \texttt{atemp}: temperatura percepita in gradi Celsius

  \item \texttt{humidity}: umidità relativa (in percentuale)

  \item \texttt{windspeed}: velocità del vento

  \end{itemize}
\item \textbf{Variabili risposta}:
    \begin{itemize}

  \item \texttt{casual}:  biciclette noleggiate da utenti non registrati

  \item \texttt{registered}: biciclette noleggiate da utenti registrati

  \item \texttt{count}: totale delle biciclette noleggiate

    \end{itemize}
\end{itemize}
