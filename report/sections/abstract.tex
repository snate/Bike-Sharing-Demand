\section{Introduzione}

\subsection{Abstract}
Il progetto consiste nell'analisi di dati riguardanti il \emph{Bike sharing}.
Il \emph{Bike sharing} è un servizio che viene fornito nelle principali città,
tramite il quale è possibile utilizzare una bici prelevandola da un chiosco
multimediale e riportandola ad un altro chiosco una volta terminato l'utilizzo.

In questo modo le persone possono spostarsi per la città in base ad esigenze
impreviste, senza aver doversi muovere con la propria bici, cosa
particolarmente scomoda se per raggiungere il posto in cui serve
effettivamente utilizzare la bici bisogna utilizzare un servizio pubblico come
autobus o treno.

Naturalmente, questi spostamenti sono molto interessanti per i \emph{data
scientist} poichè (ad esempio) forniscono dati secondari su:

\begin{itemize}
\item durata del viaggio;
\item luoghi di partenza più ricorrenti;
\item luoghi di arrivo più ricorrenti.
\end{itemize}

Questa possibilità di ottenere dati in un modo molto semplice ha spinto gli
stessi ricercatori ad analizzare la richiesta del servizio di \emph{Bike
sharing}, così da poter prevedere le possibili quantità di dati disponibili.

