\section{Script}\label{sec:script}

\subsection{populate.R}\label{sec:script-populate}
\begin{verbatim}
# Read tables
test = read.csv(file="data/test.csv")
train = read.csv(file="data/train.csv")

# DATETIME # Consider only hour
test$datetime = as.POSIXct(test$datetime)
test$datetime = strptime(test$datetime, "%Y-%m-%d %H:%M:%S")
test$datetime <- as.numeric(format(test$datetime, "%H"))
train$datetime = as.POSIXct(train$datetime)
train$datetime = strptime(train$datetime, "%Y-%m-%d %H:%M:%S")
train$datetime <- as.numeric(format(train$datetime, "%H"))

# HOURS # Treat hour of day as a qualitative covariate
test$datetime = factor(test$datetime)
train$datetime = factor(train$datetime)

# SEASONS # Treat as qualitative covariate
# summer
train$season.summer = rep(0,length(train$season))
train$season.summer[train$season == 2] = 1
test$season.summer = rep(0,length(test$season))
test$season.summer[test$season == 2] = 1
# fall
train$season.fall = rep(0,length(train$season))
train$season.fall[train$season == 3] = 1
test$season.fall = rep(0,length(test$season))
test$season.fall[test$season == 3] = 1
# bring all to 0 except spring
test$season[test$season == 2] = 0
test$season[test$season == 3] = 0
test$season[test$season == 4] = 0
train$season[train$season == 2] = 0
train$season[train$season == 3] = 0
train$season[train$season == 4] = 0

# COSTANTS # Some constants
columns = colnames(test)
FWD_SW_THRESHOLD = 0.1
MAX_P_DEGREE = 30
F1 = as.formula(paste("log(train$count)~",paste(names(train[-c(10:12)]), collapse="+")))

# LIFT-ROC # Import lift-roc script
source("scripts/lift-roc1.R")

\end{verbatim}

\subsection{linearModel.R}\label{sec:script-linear-model}

\begin{verbatim}
r2 = rep(0,length(columns)); fstatistic = rep(0,length(columns))

for(i in 1:length(columns)) {
  name = columns[i]
  r2[i] = (cov(train[name],train$count)^2) / (var(train[name]) * var(train$count))
  fstatistic[i] = r2[i] * (length(train$count) - 2) / (1 - r2[i])
  print(paste(name, "has R-squared:", r2[i], " and F-statistic: ", fstatistic[i]))
}

best = 1
for(i in 2:length(columns)) {
  if(fstatistic[i] > fstatistic[best])
    best = i
}

print(paste(columns[best], "has best F-statistic:", fstatistic[best]))
print(paste(columns[best], "will be used to calculate the linear model w/ 1 variable"))

train.lm = lm(log(train$count) ~ ., data = train[columns[best]])

already_present = c(columns[best])
source("scripts/linear_model_forward_steps.R")
print(already_present)

#clean
rm(name)
rm(i)
rm(best)
rm(r2)
rm(fstatistic)
rm(already_present)
\end{verbatim}

\subsection{linear\_model\_forward\_steps.R}
\label{sec:script-linear-model-fwd-steps}

\begin{verbatim}
bestProb = 1
#already_present = c(already_present,42)

for(i in 1:length(columns)) {
  if(columns[i] %in% already_present)
    next
  myFrame = train[,c(already_present,columns[i])]
  train.lm2 = lm(log(train$count) ~ .,data = myFrame)
  
  myAnova = anova(train.lm,train.lm2)
  print(columns[i])
  str(myAnova$`Pr(>F)`)
  if(myAnova$`Pr(>F)`[2] < bestProb) {
    bestProb = myAnova$`Pr(>F)`[2]
    best = i
  }
}

if(bestProb < FWD_SW_THRESHOLD) {
  already_present = c(already_present,columns[best])
  myFrame = train[,c(already_present)]
  train.lm = lm(log(train$count) ~ .,data = myFrame)
  print(paste(columns[best], "column has been added to our linear model"))
  source("scripts/linear_model_forward_steps.R")
}

#clean
if(exists("myFrame"))
  rm(myFrame)
if(exists("myAnova"))
  rm(myAnova)
if(exists("bestProb"))
  rm(bestProb)
\end{verbatim}

\subsection{KalmanFilter.R}\label{sec:script-kalman}
\begin{verbatim}
indexes = sample(1:NROW(train),length(train$count)-10)

Aprimo = train[indexes,columns]
y = log(train[indexes,"count"])
X = model.matrix(~., data=Aprimo)

V = diag(1,ncol(X),ncol(X))
beta = matrix( c( mean(y[1]), rep(0, ncol(X) - 1) ), ncol(X), 1 )

beta.storia = matrix(NA, nrow(X), ncol(X))
beta.storia[1,] = beta

for(i in 1 : NROW(X)) {
  H = 1 / (1 + t(X[i,]) %*% V %*% X[i,] )
  beta = beta + H[1] * (V %*% X[i,] %*% (y[i] - t(X[i,] %*% beta)) )
  V = V - H[1] * (V %*% X[i,] %*% t(X[i,]) %*% V )
  beta.storia[i,] = beta
}

beta = matrix(beta)
rownames(beta) = c("(Intercept)",columns)

X11()
par(mfrow = c(3,1))
plot(beta.storia[,1], type="l")
plot(beta.storia[,2], type="l")
plot(beta.storia[,3], type="l")

rm(Aprimo)
rm(H)
rm(V)
rm(X)
rm(i)
rm(y)
rm(indexes)
rm(beta.storia)
#rm(beta)
\end{verbatim}